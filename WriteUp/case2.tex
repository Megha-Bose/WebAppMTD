\section{Case 2}\label{sec:case2}
In this section, we go one step further to assume that we know the number of types and the probability distribution across these types but we cannot observe which type carried out the attack. However, we know that the ranges of utility obtained by each attacker type has low intersection. This is motivated by the setting where attacker types are divided based on skill and the ability to cause harm. We formalise this intuition by defining a metric we call {\em degree of overlap} and then we propose algorithms for settings with low degree of overlap.
\subsection{Degree of Overlap}\label{subsec:doo}
Even though we do not know the utilities before hand, we assume that we know a probability distribution from where defender utilities are drawn from. The distribution for each type need not be the same and we represent this distribution by $\cal D_{\Psi_t}$ for type $\Psi_t$. From this, the degree of overlap of a security game, $DoO$ is defined by 
\begin{align*}
    DoO = \frac{\sum_{\Psi_i, \Psi_j \in \Psi}AreaOfIntersection(D_{\Psi_i}, D_{\Psi_j})}{^\tau C_2}
\end{align*}
where $\tau$ is the number of types and $AreaOfIntersection$ is the area of intersection of the two probability mass functions when you plot it on a graph. Mathematically, if $f_{D_{\Psi_i}}$ and $f_{D_{\Psi_j}}$ are two continuously distributed probability mass functions of distributions $D_{\Psi_i}$ and $D_{\Psi_j}$ respectively, then
\begin{align*}
    AreaOfIntersection(D_{\Psi_i}, D_{\Psi_j}) = \int \min\{f_{D_{\Psi_i}}(x), f_{D_{\Psi_j}}(x)\} \,dx
\end{align*}
The significance of this parameter is that a game which has a lower degree of overlap should intuitively be classified better by classification algorithm, and therefore should give better performance. A game with degree of overlap 0 should cluster perfectly and give you a performance similar to that of the partial information case (Section \ref{sec:case1}). This is the intuition we use to take our algorithm one step further to create FPL-MTD-CL (see Algorithm \ref{alg:fpl-mtd-cl}).
FPL-MTD-CL is just like FPL-MTD (Algorithm \ref{alg:fpl-mtd}) but it uses a clustering algorithm to find out the type of the attacker which attacked. We use the exact same policy selection and parameter update rule.
\begin{algorithm}
\DontPrintSemicolon
$\hat{r}_{e,\psi} = 0 \quad \forall e \in N, \psi \in \Psi$\;
$\hat{s}_{c, c'} = 0 \quad \forall c, c' \in C$\;
\For{$t = 1,2,\dots, T$}{
    Sample $flag \in \{0,1\}$ such that $flag = 0$ with prob $\gamma$\;
    \uIf{$flag==0$}{
        Let $v_t$ be a randomly sampled configuration\;
    }
    \Else{
        Draw $z_{e, \psi} \gets exp(\eta)$ independently for $e \in N$ and $\psi \in \Psi$\;
        Let $u_c = \sum_{\psi \in \Psi} \min_{e \in \cal V_c} \cal P_{\psi} (\hat{r}_{e, \psi} - z_{e, \psi}) \quad \forall c \in C$\;
        Let $v_t = \max_{c \in C} e^{-g(c)} u_c - \hat{s}_{v_{t-1}, c}$\;
    }
    Adversary plays $a_t$, you play $v_t$ and get a reward $r_t(a_t, v_t)$ and incur a switching cost $s(v_{t-1}, v_t)$\;
    Run GR-Reward to estimate $\frac{1}{p(a_t \in \cal V_{v_t} \land \Psi_{f(t)})}$ as $K_r(a_t, v_t \Psi_{f(t)})$\;
    Use a clustering algorithm to classify the adversary into type $\Psi_{f(t)}$ \;
    Run GR-Switch to estimate $\frac{1}{p(v_t/ v_{t-1})}$ as $K_s(v_{t-1}, v_t)$\;
    
    Update $\hat{r}_{a_t,\Psi_{f(t)}} = \hat{r}_{a_t,\Psi_{f(t)}} + K_r(a_t, v_t, \Psi_{f(t)})r_t(a_t, \Psi_{f(t)})$\;
    
    Update $\hat{s}_{v_{t-1}, v_t} = \hat{s}_{v_{t}, v_{t-1}} = \hat{s}_{v_{t-1}, v_t} + K_s(v_{t-1}, v_t) s(v_{t-1}, v_t)$\;
}
\caption{FPL-MTD-CL}
\label{alg:fpl-mtd-cl}
\end{algorithm}

Note that a low degree of overlap is a very strong assumption and may not be true in a lot of cases. In these cases, we can use Algorithm \ref{alg:fpl-mtd} but assume there is only one attacker type. 

