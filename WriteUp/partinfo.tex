\section{Low Information}\label{sec:low-info}
In this section, we only assume that we can observe rewards at the end of each time step. We assume no knowledge about the set of vulnerabilities or the attacker which attacked at a given time step. This setting corresponds to a new application which uses much newer software whose vulnerabilities are unknown and the different types of attackers trying to exploit the software are unknown as well.


If all the attacks and the rewards are chosen adversarially, this resembles an adaptive multi-armed bandit problem with switching costs. Even when the switching costs are $0$, the lower bound on policy regret can be shown to be $\Omega(T)$. However, we propose an algorithm that does well for most realistic attacker types. This algorithm is an extension of the Exp3 algorithm [cite Auer 2002] and the main difference is that we add some padding to reduce the number of times the algorithm switches so as to minimize the number of switches. The detailed algorithm is presented in Algorithm \ref{algo:padded-exp3}.

\begin{algorithm}
\DontPrintSemicolon
$t = 1$\;
$w_c(t) = 1 \forall c \in \cal C$\;
\For{$t = 1, 2, \dots$}{
	$\eta \sim U[0, 1]$\;
	\If{$\eta < \eta'$}{
		$c_t = c_{t-1}$\;
	}
	\Else{
		$p_c(t) = (1 - \gamma) \frac{w_c(t)}{\sum_{c' \in \cal C} w_c(t)} + \frac{\gamma}{|\cal C|}$\;
		Choose $c_t$ according to the probabilities $\{p_c(t)\}_{c \in \cal C}$\; 
	}
	Receive reward, $r_t$ and switching cost $s_t$\;
	$w_c(t+1) = w_c(t)exp\{\frac{\gamma r_t}{p_c(t) |\cal C|}\}$;
}
\caption{Padded Exp3}
\label{algo:padded-exp3}
\end{algorithm}

This algorithm retains the essence of the Exp3 algorithm in the sense that it computes a probability distribution over all the configurations using some weights and chooses an action using this probability distribution. But the key difference is that this step is only done very few times to reduce the number of switches that the algorithm makes. In other words, at time $t$, the probability of choosing $c_{t-1}$ is padded to ensure that the number of switches done by the algorithm is low.