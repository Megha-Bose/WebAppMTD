%%
%% sample document for AAMAS'19 conference
%%
%% modified from sample-sigconf.tex
%%
%% see ACM instructions acmguide.pdf
%%
%% AAMAS-specific questions? F.A.Oliehoek@tudelft.nl
%%

\documentclass[sigconf]{aamas}  % do not change this line!

%% your usepackages here, for example:
\usepackage{booktabs}

%% do not change the following lines
\usepackage{flushend}
\usepackage{amsmath}
\usepackage[ruled]{algorithm2e}
\setcopyright{ifaamas}  % do not change this line!
\acmDOI{doi}  % do not change this line!
\acmISBN{}  % do not change this line!
\acmConference[AAMAS'20]{Proc.\@ of the 19th International Conference on Autonomous Agents and Multiagent Systems (AAMAS 2020), B.~An, N.~Yorke-Smith, A.~El~Fallah~Seghrouchni, G.~Sukthankar (eds.)}{May 2020}{Auckland, New Zealand}  % do not change this line!
\acmYear{2020}  % do not change this line!
\copyrightyear{2020}  % do not change this line!
\acmPrice{}  % do not change this line!

%% the rest of your preamble here
\newcommand{\cal}[1]{\mathcal{#1}}
\newcommand{\norm}[1]{\left\lVert{#1}\right\rVert}


\if 0
\newtheorem{theorem}{Theorem}[section]
\newtheorem{lemma}[theorem]{Lemma}
\newtheorem{conj}[theorem]{Conjecture}
\newtheorem{claim}[theorem]{Claim}
\newtheorem{prop}[theorem]{Proposition}

\theoremstyle{definition}
\newtheorem{remark}[theorem]{Remark}

\theoremstyle{definition}
\newtheorem{definition}[theorem]{Definition}
\newtheorem{obs}[theorem]{Observation}
\newtheorem{example}[theorem]{Example}
\fi

%%%%%%%%%%%%%%%%%%%%%%%%%%%%%%%%%%%%%%%%%%%%%%%%%%%%%%%%%%%%%%%%%%%%%%%%%%%%%%%%%%%%%%%%%%%%%%%%%%%%%%%%%

\begin{document}

\title{Moving Target Defense under Uncertainty for Web Applications}  % put your title here!
%\titlenote{Produces the permission block, and copyright information}

% AAMAS: as appropriate, uncomment one subtitle line; check the CFP
%\subtitle{Extended Abstract}
%\subtitle{Blue Sky Ideas Track}
%\subtitle{JAAMAS Track}
%\subtitle{Demonstration}
%\subtitle{Doctoral Consortium}

% AAMAS: submissions are anonymous for most tracks
\author{Paper \#XXX}  % put your paper number here!

%% example of author block for camera ready version of accepted papers: don't use for anonymous submissions
%
%\author{Ben Trovato}
%\authornote{Dr.~Trovato insisted his name be first.}
%\orcid{1234-5678-9012}
%\affiliation{%
%  \institution{Institute for Clarity in Documentation}
%  \streetaddress{P.O. Box 1212}
%  \city{Dublin} 
%  \state{Ohio} 
%  \postcode{43017-6221}
%}
%\email{trovato@corporation.com}
%
%\author{G.K.M. Tobin}
%\authornote{The secretary disavows any knowledge of this author's actions.}
%\affiliation{%
%  \institution{Institute for Clarity in Documentation}
%  \streetaddress{P.O. Box 1212}
%  \city{Dublin} 
%  \state{Ohio} 
%  \postcode{43017-6221}
%}
%\email{webmaster@marysville-ohio.com}
%
%\author{Lars Th{\o}rv{\"a}ld}
%\authornote{This author is the
%  one who did all the really hard work.}
%\affiliation{%
%  \institution{The Th{\o}rv{\"a}ld Group}
%  \streetaddress{1 Th{\o}rv{\"a}ld Circle}
%  \city{Hekla} 
%  \country{Iceland}}
%\email{larst@affiliation.org}
%
%\author{Valerie B\'eranger}
%\affiliation{%
%  \institution{Inria Paris-Rocquencourt}
%  \city{Rocquencourt}
%  \country{France}
%}
%\author{Aparna Patel} 
%\affiliation{%
% \institution{Rajiv Gandhi University}
% \streetaddress{Rono-Hills}
% \city{Doimukh} 
% \state{Arunachal Pradesh}
% \country{India}}
%\author{Huifen Chan}
%\affiliation{%
%  \institution{Tsinghua University}
%  \streetaddress{30 Shuangqing Rd}
%  \city{Haidian Qu} 
%  \state{Beijing Shi}
%  \country{China}
%}
%
%\author{Charles Palmer}
%\affiliation{%
%  \institution{Palmer Research Laboratories}
%  \streetaddress{8600 Datapoint Drive}
%  \city{San Antonio}
%  \state{Texas} 
%  \postcode{78229}}
%\email{cpalmer@prl.com}
%
%\author{John Smith}
%\affiliation{\institution{The Th{\o}rv{\"a}ld Group}}
%\email{jsmith@affiliation.org}
%
%\author{Julius P.~Kumquat}
%\affiliation{\institution{The Kumquat Consortium}}
%\email{jpkumquat@consortium.net}
%
%% The example's default list of authors is too long for headers
%\renewcommand{\shortauthors}{B. Trovato et al.}

\begin{abstract}  % put your abstract here!
Moving target defense (MTD) has emerged as a technique that can be used in various applications to reduce the threat of attackers by taking away their ability to perform reconnaissance. However, a majority of the existing research in the field assumes unrealistic access to information about attacker motivations when developing MTD strategies. Many of the existing approaches also assume complete knowledge regarding the vulnerabilities of a particular application and how each of these vulnerabilities can be exploited by an attacker. In this work, we aim to create algorithms that generate effective moving target defense strategies that do not rely on prior knowledge. Our work assumes the only information we get is via interaction with the attacker in a repeated game setting. Since the amount of information that can be obtained through interactions may vary from application to application, we provide different algorithms that account for the different levels of information to identify optimal switching strategies. We then evaluate our algorithms using Common Vulnerabilities and Exploits mined from the National Vulnerability Database to show that our algorithms significantly outperform the state of the art.
\end{abstract}


\keywords{Moving Target Defense; Online Learning; Information Uncertainty}  % put your semicolon-separated keywords here!

\maketitle


%%%%%%%%%%%%%%%%%%%%%%%%%%%%%%%%%%%%%%%%%%%%%%%%%%%%%%%%%%%%%%%%%%%%%%%%%%%%%%%%%%%%%%%%%%%%%%%%%%%%%%%%%
%% start of main body of paper

\section{Introduction}\label{sec:intro}
\textbf{Add Later}

\textcolor{red}{Lets add something quick here so we get some structure}

\subsection{Our Contribution}\label{subsec:contrib}
\textbf{Add Later.}

\subsection{Related Work}\label{subsec:relwork}
\textbf{Add Later.}

\section{Model and Preliminaries}\label{sec:model}
We consider a repeated game played with two players - a defender (denoted by $\Theta$) and $\tau$ attackers (denoted by $ \Psi = \{\Psi_1, \Psi_2, \dots, \Psi_\tau\}$). At each round (or timestep) of this game, one of these attackers tries to exploit the application set up by the defender. We assume there exists a probability distribution $\cal P$ across these attacker types that determines which attacker attacks at a given time step but this distribution may not always be known to the defender. The defender has a set $\cal C = \{c_1, c_2, \dots c_n\}$ of $n$ configurations that it can deploy. Each configuration has a set of vulnerabilities $\cal V_c = \{e_1, e_2, \dots, e_{|\cal V_c|}\}$ that can be exploited. We define the set of all vulnerabilities by  $N = \bigcup_{c \in C} \cal V_c$. This vulnerability set for each configuration may not be known before hand but we assume that no configuration is perfect i.e. every configuration has some vulnerability. 

At the $t$'th round, the defender chooses a configuration to deploy (denoted by $v_t \in \cal C$) and the attacker of type $f(t)$ (where $f$ is a function that maps a round to the attacker type at that round) chooses a vulnerability to exploit (denoted by $a_t$); the attacker can also choose not to exploit any vulnerability during a turn. Furthermore, we assume that there is a cost incurred when switching between configurations which may not be known a priori but remains constant throughout. The cost incurred by switching from configuration $c$ to configuration $c'$ in the $t$'th round is denoted by $s(c, c') \in (0,1]$. The game is played for $T$ rounds. 

For each vulnerability $e \in N$, for each configuration $c \in \cal C$, for each attacker type $\psi \in \Psi$, we define a reward to the defender at the $t$'th round denoted by $r_t(\psi, e, c)$. The reward $r_t(\psi, e, c) \in [-1, 0)$, if an attacker successfully exploits a vulnerability that the defender's configuration has i.e. $e \in \cal V_c$ and is $0$ otherwise. Note that the different attacker types will result in the defender obtaining different utilities because of the varying attacker capabilities e.g., some attackers may not have the expertise to carry out certain attacks \citep{sailik2016webappmtd}. It is important to note that we do not require the rewards to be constant throughout; they can be stochastic or even adversarial in nature.

We make no assumptions about the attacker strategy since the attacker may not be fully rational. We also do not make assumptions about the attacker utility, since these values will be hard to observe and even harder to find out beforehand. We however assume that the attacker has access to information from the previous rounds and is capable of reconnaissance.

Our goal in this paper is to maximise the total utility that the defender receives, i.e.
\begin{align*}
    \sum_{t=1}^T r_t(\Psi_{f(t)}, a_t, v_t) - s_t(v_{t-1}, v_t)
\end{align*}

\section{High Information}\label{sec:high-info}
In this section, we consider a scenario where the defender fully knows the vulnerabilities of every configuration that they deploy, the number of attackers and the reward that the defender would get when an attacker attacks a vulnerability. At every time step, we observe what vulnerability was attacked, which attacker attacked it and the reward obtained. The only two things the defender does not know is the strategy of the attacker and the probability distribution across attacker types. Since we do not know the probability distribution among the attackers, we assume the attackers are chosen adversarially. This setting corresponds to an older application where enough research has been done to determine the vulnerabilities of each configuration that is being deployed and past data about attacks has helped determine the different kinds of attackers who are trying to exploit vulnerabilities in this software and the damage they can do.

When the rewards $r(\psi, e, c)$ are constant for each attacker, vulnerability and configuration tuple, then we can model this as a repeated game against a {\em master} attacker who chooses both an attacker and a vulnerability. \textbf{Need to figure this out}.



\section{Low Information}\label{sec:low-info}
In this section, we only assume that we can observe rewards at the end of each time step. We assume no knowledge about the set of vulnerabilities or the attacker which attacked at a given time step. This setting corresponds to a new application which uses much newer software whose vulnerabilities are unknown and the different types of attackers trying to exploit the software are unknown as well.


If all the attacks and the rewards are chosen adversarially, this resembles an adaptive multi-armed bandit problem with switching costs. Even when the switching costs are $0$, the lower bound on policy regret can be shown to be $\Omega(T)$. However, we propose an algorithm that does well for most realistic attacker types. This algorithm is an extension of the Exp3 algorithm [cite Auer 2002] and the main difference is that we add some padding to reduce the number of times the algorithm switches so as to minimize the number of switches. The detailed algorithm is presented in Algorithm \ref{algo:padded-exp3}.

\begin{algorithm}
\DontPrintSemicolon
$t = 1$\;
$w_c(t) = 1 \forall c \in \cal C$\;
\For{$t = 1, 2, \dots$}{
	$\eta \sim U[0, 1]$\;
	\If{$\eta < \eta'$}{
		$c_t = c_{t-1}$\;
	}
	\Else{
		$p_c(t) = (1 - \gamma) \frac{w_c(t)}{\sum_{c' \in \cal C} w_c(t)} + \frac{\gamma}{|\cal C|}$\;
		Choose $c_t$ according to the probabilities $\{p_c(t)\}_{c \in \cal C}$\; 
	}
	Receive reward, $r_t$ and switching cost $s_t$\;
	$w_c(t+1) = w_c(t)exp\{\frac{\gamma r_t}{p_c(t) |\cal C|}\}$;
}
\caption{Padded Exp3}
\label{algo:padded-exp3}
\end{algorithm}

This algorithm retains the essence of the Exp3 algorithm in the sense that it computes a probability distribution over all the configurations using some weights and chooses an action using this probability distribution. But the key difference is that this step is only done very few times to reduce the number of switches that the algorithm makes. In other words, at time $t$, the probability of choosing $c_{t-1}$ is padded to ensure that the number of switches done by the algorithm is low.


\section{Experiments}\label{sec:expts}
\textbf{Add Later}

\subsection{Identifying Critical Vulnerabilities}

\section{Conclusion}\label{sec:conclusion}
\textbf{Add Later}




%%%%%%%%%%%%%%%%%%%%%%%%%%%%%%%%%%%%%%%%%%%%%%%%%%%%%%%%%%%%%%%%%%%%%%%%%%%%%%%%%%%%%%%%%%%%%%%%%%%%%%%%%
%% bibliography: see CFP for number of permitted pages

\bibliographystyle{ACM-Reference-Format}  % do not change this line!
\bibliography{sample-bibliography}  % put name of your .bib file here

\end{document}
